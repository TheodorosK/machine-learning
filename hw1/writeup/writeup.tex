\documentclass[11pt, fleqn, titlepage]{article}

\documentclass[11pt, fleqn]{article}

\usepackage[usenames,dvipsnames,svgnames,table]{xcolor}
\usepackage{amsmath}
\usepackage{amsfonts}
\usepackage[margin=1in]{geometry} % To set the margin widths
\usepackage{graphicx}
\usepackage{listings}
\usepackage{multirow}
\usepackage{tabularx}
\usepackage{varioref}
\usepackage[noabbrev,capitalize]{cleveref}
\usepackage[group-separator={,}]{siunitx}
\usepackage{subcaption}
\usepackage{titlesec}
\usepackage{lscape}
\usepackage{bm}
\usepackage[titletoc,toc,title]{appendix}

\lstset{
  frame=single,
  basicstyle=\ttfamily,% print whole listing small
  language=R,
  aboveskip=3mm,
  belowskip=3mm,
  showstringspaces=false,
  columns=flexible,
  numbers=none,
  commentstyle=\color{ForestGreen},
  stringstyle=\color{Maroon},
  breaklines=true,
  breakatwhitespace=true,
  tabsize=2,
  literate={<-}{{$\gets$}}1 {~}{{$\sim$}}1
}

\sisetup{output-exponent-marker=\textsc{e}}

\setlength{\parskip}{12pt} % Sets a blank line in between paragraphs
\setlength\parindent{0pt} % Sets the indent for each paragraph to zero

% \crefname{figure}{Figure}{Figures}
% \crefname{section}{Section}{Sections}
% \crefname{table}{Table}{Tables}
% \crefname{lstlisting}{Listing}{Listings}

\setlength{\parskip}{12pt} % Sets a blank line in between paragraphs
\setlength\parindent{0pt} % Sets the indent for each paragraph to zero

\begin{document}

\title{Machine Learning () HW #1}
\author{Will Clark \& Matthew DeLio \\
University of Chicago Booth School of Business}
\date{\today}
\maketitle

% Describe the data, show a plot of mileage vs price

The data set for this exercise the sale price and observed mileage for 1000 used cars. We can see in Figure ???? that the expected inverse relationship between mileage and sale price is borne out by the data (i.e. high-mileage cars are less expensive than low-mileage cars).

To describe this data, we first estimate a linear model of mileage on price:

\[ p_i = \alpha + \beta m_i + \varepsilon_i \]

where \(p_i\) is the price of car \(i\), \(m_i\) is the observed mileage on car \(i\), \(\alpha\) and \(\beta\) are estimated regression coefficients, and \(\varepsilon\) is the residual.

% Linear regression

% knn
% use in-sample RMSE
% use n-fold cross-validation

% Predict car price with cross-validated knn and with linear model

\end{document}

% \input{.tex}

% \begin{figure}
%   \centering
%   \begin{subfigure}[b]{0.49\textwidth}
%     \includegraphics[width=\textwidth]{.pdf}
%     \caption{}
%     \label{fig:}
%   \end{subfigure}
%   \hfill
%   \begin{subfigure}[b]{0.49\textwidth}
%     \includegraphics[width=\textwidth]{.pdf}
%     \caption{}
%     \label{fig:}
%   \end{subfigure}
%   \caption{}
% \end{figure}

% \begin{figure}[!htb]
%   \centering
%   \includegraphics[scale=.5]{.pdf}
%   \caption{}
%   \label{fig:}
% \end{figure}

