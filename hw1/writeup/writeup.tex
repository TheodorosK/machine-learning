\documentclass[11pt, fleqn, titlepage]{article}

\usepackage{amsmath}
\usepackage{amsfonts}
\usepackage[margin=1in]{geometry} % To set the margin widths
\usepackage{graphicx}
\usepackage{listings}
\usepackage{multirow}
\usepackage{tabularx}
\usepackage{varioref}
\usepackage[noabbrev,capitalize]{cleveref}
\usepackage[group-separator={,}]{siunitx}
\usepackage{subcaption}
\usepackage{titlesec}
\usepackage{bm}

\usepackage{float}
\floatstyle{plaintop}

% \usepackage{hyperref}

% \lstset{
%   frame=top,frame=bottom,
%   basicstyle=\small\normalfont\sffamily,    % the size of the fonts that are used for the code
%   stepnumber=1,                           % the step between two line-numbers. If it is 1 each line will be numbered
%   numbersep=10pt,                         % how far the line-numbers are from the code
%   tabsize=2,                              % tab size in blank spaces
%   extendedchars=true,                     %
%   breaklines=true,                        % sets automatic line breaking
%   captionpos=b,                           % sets the caption-position to top
%   mathescape=true,
%   stringstyle=\color{white}\ttfamily, % Farbe der String
%   showspaces=false,           % Leerzeichen anzeigen ?
%   showtabs=false,             % Tabs anzeigen ?
%   xleftmargin=17pt,
%   framexleftmargin=17pt,
%   framexrightmargin=17pt,
%   framexbottommargin=5pt,
%   framextopmargin=5pt,
%   showstringspaces=false      % Leerzeichen in Strings anzeigen ?
% }


\lstset{
  captionpos=b,
  frame=single,
  language=R,
  literate = {<-}{{$\gets$}}1 {~}{{$\sim$}}1
}

\sisetup{output-exponent-marker=\textsc{e}}

\titleformat{\section}[block]{\bfseries}{\thesection}{1em}{}

% \crefname{figure}{Figure}{Figures}
% \crefname{section}{Section}{Sections}
% \crefname{table}{Table}{Tables}
% \crefname{lstlisting}{Listing}{Listings}

\setlength{\parskip}{12pt} % Sets a blank line in between paragraphs
\setlength\parindent{0pt} % Sets the indent for each paragraph to zero

\begin{document}

\title{Machine Learning () HW #1}
\author{Will Clark \& Matthew DeLio \\
University of Chicago Booth School of Business}
\date{\today}
\maketitle

% Describe the data, show a plot of mileage vs price

The data set for this exercise the sale price and observed mileage for 1000 used cars. We can see in \cref{fig:linear} that the expected inverse relationship between mileage and sale price is borne out by the data (i.e. high-mileage cars are less expensive than low-mileage cars).

To describe this data, we first estimate a linear model of mileage on price:
\[ p_i = \alpha + \beta m_i + \varepsilon_i \]
where \(p_i\) is the price of car \(i\), \(m_i\) is the observed mileage on car \(i\), \(\alpha\) and \(\beta\) are estimated regression coefficients, and \(\varepsilon_i\) is the residual. We fit a generalized linear model to the data and find that \(\alpha=56359.7\) and \(\beta=-0.35\). That is, a hypothetical used car with no miles would sell for about \$56 thousand, and each additional mile driven lowers a car's expected selling price by \$0.35.

As depicted in \cref{fig:linear}, this model appears to be a weak fit for the data. The distribution of residuals is not normally distributed. The model tends to systematically under-value cars with very low mileage and with very high mileage (see \cref{fig:lin_errors} in the Appendix). 

\begin{figure}[!htb]
  \centering
  \includegraphics[scale=.5]{linear_fit.pdf}
  \caption{Linear Regression of Price on Mileage}
  \label{fig:linear}
\end{figure}

As an alternative to a linear model, 

\section{kNN}
% knn
% use in-sample RMSE
% use n-fold cross-validation

\begin{figure}[!htb]
  \centering
  \includegraphics[scale=.5]{sweep_kknn.pdf}
  \caption{CV and OOS RMSE for Various Values of k in kNN}
  \label{fig:sweep}
\end{figure}


% Predict car price with cross-validated knn and with linear model

\section{Appendix}

\begin{figure}[!htb]
  \centering
  \includegraphics[scale=.5]{lin_errors.pdf}
  \caption{Distribution of Residuals in Linear Model}
  \label{fig:lin_errors}
\end{figure}

\end{document}

% \input{.tex}

% \begin{figure}
%   \centering
%   \begin{subfigure}[b]{0.49\textwidth}
%     \includegraphics[width=\textwidth]{.pdf}
%     \caption{}
%     \label{fig:}
%   \end{subfigure}
%   \hfill
%   \begin{subfigure}[b]{0.49\textwidth}
%     \includegraphics[width=\textwidth]{.pdf}
%     \caption{}
%     \label{fig:}
%   \end{subfigure}
%   \caption{}
% \end{figure}

% \begin{figure}[!htb]
%   \centering
%   \includegraphics[scale=.5]{.pdf}
%   \caption{}
%   \label{fig:}
% \end{figure}

