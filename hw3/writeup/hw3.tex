\documentclass[11pt, fleqn]{article}

\documentclass[11pt, fleqn]{article}

\usepackage[usenames,dvipsnames,svgnames,table]{xcolor}
\usepackage{amsmath}
\usepackage{amsfonts}
\usepackage[margin=1in]{geometry} % To set the margin widths
\usepackage{graphicx}
\usepackage{listings}
\usepackage{multirow}
\usepackage{tabularx}
\usepackage{varioref}
\usepackage[noabbrev,capitalize]{cleveref}
\usepackage[group-separator={,}]{siunitx}
\usepackage{subcaption}
\usepackage{titlesec}
\usepackage{lscape}
\usepackage{bm}
\usepackage[titletoc,toc,title]{appendix}

\lstset{
  frame=single,
  basicstyle=\ttfamily,% print whole listing small
  language=R,
  aboveskip=3mm,
  belowskip=3mm,
  showstringspaces=false,
  columns=flexible,
  numbers=none,
  commentstyle=\color{ForestGreen},
  stringstyle=\color{Maroon},
  breaklines=true,
  breakatwhitespace=true,
  tabsize=2,
  literate={<-}{{$\gets$}}1 {~}{{$\sim$}}1
}

\sisetup{output-exponent-marker=\textsc{e}}

\setlength{\parskip}{12pt} % Sets a blank line in between paragraphs
\setlength\parindent{0pt} % Sets the indent for each paragraph to zero

% \crefname{figure}{Figure}{Figures}
% \crefname{section}{Section}{Sections}
% \crefname{table}{Table}{Tables}
% \crefname{lstlisting}{Listing}{Listings}

\setlength{\parskip}{12pt} % Sets a blank line in between paragraphs
\setlength\parindent{0pt} % Sets the indent for each paragraph to zero

\begin{document}

\title{Machine Learning (41204-01)\\HW \#3}
\author{Will Clark $\vert$ Matthew DeLio \\
\texttt{will.clark@chicagobooth.edu} $\vert$ \texttt{mdelio@chicagobooth.edu} \\
University of Chicago Booth School of Business}
\date{\today}
\maketitle

\section{Data}

For this exercise, our data set contains the sale price and the observable characteristics for a sample of 20,000 used cars. We break this data into three subsets: 50 percent of data will be our training set that will be used to train/tune our models ($n=10,031$); 25 percent will be our validation data set which we will use to evaluate model performance ($n=5,016$); and 25 percent will be our test set which we will use to evaluate out-of-sample performance of our best model ($n=5,016$).

We will build a series of models that can predict the selling price of a used car given its observable characteristics. The models and techniques we will use are: (1) regression trees, (2) bagging, (3) random forests, and (4) boosting trees.

\section{Regression Trees}



\section{Bagging}

In this section, we use an aggregate bootstrap technique to predict used car sales price. The basic algorithm is:
\begin{itemize}
\item For a given number of trials $T$:
\begin{itemize}
\item Select a bootstrap sample from the data and fit a large regression tree on this sample (in this case we take large to mean a tree that is not pruned;
\item Use the large tree to make a prediction for expected price;
\end{itemize}
\item Take an average of the predicted price across all trials.
\end{itemize}

Ultimately, the bagging algorithm does not perform very well relative to the boosting tree, the random forest, and the LASSO regression. 

\section{Random Forest}

In this section, we try to predict car price with a random forest algorithm. The main difference between the algorithm here and that in the prior section is that for each tree, instead of estimating based on the entire set of covariates, we estimate only on a subset $m$ of covariates. This makes the algorithm run more quickly and introduces another layer of randomness into our predictions. 

We chose a value of $m=3$, which produced the best set of out-of-sample RMSE values. We also chose to stop the algorithm after 250 trees, as the predictive performance (measured by out-of-bag RMSE) fails to improve after this point. In order to speed up performance even more, we can also divide our intended number of trees (250) by the number of processors available (eight, in this case). We then build each small forest on a separate processor and combine the eight forests at the end into one larger forest and use this final combined forest to predict.

The random forest algorithm ends up performing very well, nearly beating the best-in-class performance of the boosting tree.

\section{Boosting Trees}

\section{Comparison and Out-of-Sample Test}

In \cref{tab:rmse_comp}, we list the out-of-sample RMSE of each model on the validate data set. The models denoted by ``large'' are those trained on the entire set of covariates; those denoted by ``small'' are trained only on the mileage series.

The best performing model is the large boosting tree model, followed very closely by the large random forest model and the large LASSO regression model. All three models perform similarly, although the boosting tree model was the most difficult to tune.

% latex table generated in R 3.2.1 by xtable 1.7-4 package
% Thu Oct 15 03:45:24 2015
\begin{table}[ht]
\centering
\caption{Comparison of RMSE for Various Models} 
\label{tab:rmse_comp}
\begin{tabular}{lr}
  \hline
Model & RMSE \\ 
  \hline
Boosting Tree (big) & 2895.43 \\ 
  Random Forest (big) & 2965.92 \\ 
  LASSO Regression (big) & 3200.54 \\ 
  Bagging (big) & 3407.65 \\ 
  Regression Tree (big) & 3407.65 \\ 
  Pruned Regression Tree (big) & 4588.53 \\ 
  Boosting Tree (small) & 7248.59 \\ 
  Bagging (small) & 7292.48 \\ 
  Regression Tree (small) & 7292.48 \\ 
  Pruned Regression Tree (small) & 7512.58 \\ 
  Random Forest (small) & 8389.08 \\ 
   \hline
\end{tabular}
\end{table}


\end{document}