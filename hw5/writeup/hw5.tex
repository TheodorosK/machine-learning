\usepackage{amsmath}
\usepackage{amsfonts}
\usepackage[margin=1in]{geometry} % To set the margin widths
\usepackage{graphicx}
\usepackage{listings}
\usepackage{multirow}
\usepackage{tabularx}
\usepackage{varioref}
\usepackage[noabbrev,capitalize]{cleveref}
\usepackage[group-separator={,}]{siunitx}
\usepackage{subcaption}
\usepackage{titlesec}
\usepackage{bm}

\usepackage{float}
\floatstyle{plaintop}

% \usepackage{hyperref}

% \lstset{
%   frame=top,frame=bottom,
%   basicstyle=\small\normalfont\sffamily,    % the size of the fonts that are used for the code
%   stepnumber=1,                           % the step between two line-numbers. If it is 1 each line will be numbered
%   numbersep=10pt,                         % how far the line-numbers are from the code
%   tabsize=2,                              % tab size in blank spaces
%   extendedchars=true,                     %
%   breaklines=true,                        % sets automatic line breaking
%   captionpos=b,                           % sets the caption-position to top
%   mathescape=true,
%   stringstyle=\color{white}\ttfamily, % Farbe der String
%   showspaces=false,           % Leerzeichen anzeigen ?
%   showtabs=false,             % Tabs anzeigen ?
%   xleftmargin=17pt,
%   framexleftmargin=17pt,
%   framexrightmargin=17pt,
%   framexbottommargin=5pt,
%   framextopmargin=5pt,
%   showstringspaces=false      % Leerzeichen in Strings anzeigen ?
% }


\lstset{
  captionpos=b,
  frame=single,
  language=R,
  literate = {<-}{{$\gets$}}1 {~}{{$\sim$}}1
}

\sisetup{output-exponent-marker=\textsc{e}}

\titleformat{\section}[block]{\bfseries}{\thesection}{1em}{}

\begin{document}

\title{Machine Learning (41204-01)\\HW \#5}
\author{Will Clark $\vert$ Matthew DeLio \\
\texttt{\{will.clark,mdelio\}@chicagobooth.edu} \\
University of Chicago Booth School of Business}
\date{\today}
\maketitle

\section{Baseline Predictive Models} \label{baseline}

We use a multinomial logistic regression, a random forest, and a boosting tree as a set of baseline models/algorithms against which we can assess the performance of the neural networks discussed in \cref{nnets}. 

First, we use a multinomial logistic regression model with an L1 penalty for variable selection/reduction. We use five-fold cross validation to select the optimal regularization parameter for each classification (i.e. there are six logistic regression models, each with its own L1 penalty). On our test data set, we find that this model predicts movement type with 95.4 percent accuracy. We can see in \cref{fig:heatmap_dmr} or \cref{tab:conmat_dmr} that the model predicts extremely well overall. The most common mistake it makes is to confuse standing/sitting, and it sometimes has trouble distinguishing between walking vs. walking up or down stairs.\footnote{We find the heatmap representation of the confusion matrix to be a quick and helpful way to visualise a model's predictive accuracy. Full tabular results can be found in the appendix.}

Next, we try a random forest algorithm. The only parameter to tune for a random forest is the number of covariates sampled at each tree split (\texttt{mtry}). A rule of thumb is to set $\texttt{mtry}=\sqrt{p}$. In this case, $p=477$, so we estimated a random forest for each of $\texttt{mtry}\in(10, 15, 20, 25, 30)$ so that the range would be roughly centered around $\sqrt{p}$. Again using five-fold cross validation, we found the optimal \texttt{mtry} to be 10. This model predicts out of sample with 94.0 percent accuracy, making it slightly less effective than our multinomial logit model above.\footnote{In sample accuracy was above 98 percent, although the training and test set are composed of different groups of people, so we should not be surprised to see lower OOS accuracy.} We can see in \cref{fig:heatmap_rf} or \cref{tab:conmat_rf} that this algorithm, like the one discussed above, also confuses sitting and standing. It also tends to get a bit more confused between walking vs. walking up/down stairs than the multinomial logit model.

Lastly, we try a boosting tree algorithm. The parameters we can use to tune and the ranges over which we sampled are:
\begin{itemize}
\item \texttt{interaction.depth} $\in (1, 5, 9)$
\item \texttt{n.trees} $\in (500, 1000, 1500, 2000)$
\item \texttt{shrinkage} $\in (0.01, 0.05)$
\end{itemize}
The optimal algorithm, again chosen by five-fold cross validation, used 2000 trees, interaction depth of 5, and a shrinkage parameter of 0.05. This algorithm predicts out of sample with 94.4 percent accuracy, making it slightly more accurate than the random forest but still not as accurate as our multinomial logit model. We can see in \cref{fig:heatmap_boost} or \cref{tab:conmat_boost} that this algorithm is worse than the two prior at distinguishing between sitting/standing, but it does slightly better than the random forest at distinguishing between walking vs. walking on stairs.

\begin{landscape}
\begin{figure}
  \centering
  \begin{subfigure}[b]{0.45\textwidth}
    \caption{Confusion Heatmap: Multinomial Logit}
    \includegraphics[width=\textwidth]{heatmap_dmr.pdf}
    \label{fig:heatmap_dmr}
  \end{subfigure}
  \hfill
  \begin{subfigure}[b]{0.45\textwidth}
    \caption{Confusion Heatmap: Random Forest}
    \includegraphics[width=\textwidth]{heatmap_rf.pdf}
    \label{fig:heatmap_rf}
  \end{subfigure}
  \hfill
    \begin{subfigure}[b]{0.45\textwidth}
    \caption{Confusion Heatmap: Boosting Tree}
    \includegraphics[width=\textwidth]{heatmap_boost.pdf}
    \label{fig:heatmap_boost}
  \end{subfigure}
  \caption{Confusion Matrix Heatmaps for Baseline Predictive Models}
\end{figure}
\end{landscape}


\section{Neural Nets} \label{nnets}

% latex table generated in R 3.2.1 by xtable 1.7-4 package
% Wed Nov  4 23:28:56 2015
\begin{table}[ht]
\centering
\caption{Confusion Matrix for Neural Network} 
\label{tab:nnet}
\begin{tabular}{l|rrrrrr}
  &\multicolumn{6}{c}{Reference}\\
 \hline
Predicted & Walking & WalkingUpstairs & WalkingDownstairs & Sitting & Standing & Laying \\ 
  \hline
Walking & 492 &  18 &   4 &   0 &   0 &   0 \\ 
  WalkingUpstairs &   0 & 430 &  14 &   0 &   0 &   0 \\ 
  WalkingDownstairs &   4 &  22 & 402 &   0 &   0 &   0 \\ 
  Sitting &   0 &   0 &   0 & 430 &  60 &   0 \\ 
  Standing &   0 &   1 &   0 &  58 & 472 &   0 \\ 
  Laying &   0 &   0 &   0 &   3 &   0 & 537 \\ 
   \hline
\end{tabular}
\end{table}



\clearpage
\begin{appendices}

\section{Confusion Matrices for Baseline Models}

% latex table generated in R 3.2.2 by xtable 1.7-4 package
% Wed Nov  4 17:21:32 2015
\begin{table}[ht]
\centering
\caption{Confusion Matrix for Multinomial Logit Regression} 
\label{tab:conmat_dmr}
\begin{tabular}{rrrr}
  \hline
 & Sensitivity & Specificity & Balanced Accuracy \\ 
  \hline
Class: Walking & 1.00 & 0.99 & 0.99 \\ 
  Class: WalkingUpstairs & 0.93 & 0.99 & 0.96 \\ 
  Class: WalkingDownstairs & 0.96 & 1.00 & 0.98 \\ 
  Class: Sitting & 0.88 & 0.99 & 0.93 \\ 
  Class: Standing & 0.95 & 0.98 & 0.96 \\ 
  Class: Laying & 1.00 & 1.00 & 1.00 \\ 
   \hline
\end{tabular}
\end{table}

% latex table generated in R 3.2.2 by xtable 1.7-4 package
% Wed Nov  4 20:30:16 2015
\begin{table}[ht]
\centering
\caption{Model Statistics for Multinomial Logit Regression} 
\label{tab:conmat_stats_dmr}
\begin{tabular}{rrrr}
  \hline
 & Sensitivity & Specificity & Balanced Accuracy \\ 
  \hline
Class: Walking & 1.000 & 0.985 & 0.993 \\ 
  Class: WalkingUpstairs & 0.928 & 0.994 & 0.961 \\ 
  Class: WalkingDownstairs & 0.962 & 0.999 & 0.981 \\ 
  Class: Sitting & 0.880 & 0.989 & 0.934 \\ 
  Class: Standing & 0.947 & 0.976 & 0.962 \\ 
  Class: Laying & 1.000 & 1.000 & 1.000 \\ 
   \hline
\end{tabular}
\end{table}


% latex table generated in R 3.2.2 by xtable 1.7-4 package
% Wed Nov  4 19:31:53 2015
\begin{table}[ht]
\centering
\caption{Confusion Matrix for Random Forest} 
\label{tab:conmat_rf}
\begin{tabular}{l|rrrrrr}
  &\multicolumn{6}{c}{Reference}\\
 \hline
Predicted & Walking & WalkingUpstairs & WalkingDownstairs & Sitting & Standing & Laying \\ 
  \hline
Walking & 481 &  26 &  20 &   0 &   0 &   0 \\ 
  WalkingUpstairs &   5 & 439 &  47 &   0 &   0 &   0 \\ 
  WalkingDownstairs &  10 &   6 & 353 &   0 &   0 &   0 \\ 
  Sitting &   0 &   0 &   0 & 447 &  20 &   0 \\ 
  Standing &   0 &   0 &   0 &  44 & 512 &   0 \\ 
  Laying &   0 &   0 &   0 &   0 &   0 & 537 \\ 
   \hline
\end{tabular}
\end{table}

% latex table generated in R 3.2.2 by xtable 1.7-4 package
% Wed Nov  4 19:31:53 2015
\begin{table}[ht]
\centering
\caption{Model Statistics for Random Forest} 
\label{tab:conmat_stats_rf}
\begin{tabular}{rrrr}
  \hline
 & Sensitivity & Specificity & Balanced Accuracy \\ 
  \hline
Class: Walking & 0.97 & 0.98 & 0.98 \\ 
  Class: WalkingUpstairs & 0.93 & 0.98 & 0.96 \\ 
  Class: WalkingDownstairs & 0.84 & 0.99 & 0.92 \\ 
  Class: Sitting & 0.91 & 0.99 & 0.95 \\ 
  Class: Standing & 0.96 & 0.98 & 0.97 \\ 
  Class: Laying & 1.00 & 1.00 & 1.00 \\ 
   \hline
\end{tabular}
\end{table}


% latex table generated in R 3.2.2 by xtable 1.7-4 package
% Wed Nov  4 19:31:53 2015
\begin{table}[ht]
\centering
\caption{Confusion Matrix for Boosting Tree} 
\label{tab:conmat_boost}
\begin{tabular}{l|rrrrrr}
  &\multicolumn{6}{c}{Reference}\\
 \hline
Predicted & Walking & WalkingUpstairs & WalkingDownstairs & Sitting & Standing & Laying \\ 
  \hline
Walking & 496 &  32 &   4 &   0 &   0 &   0 \\ 
  WalkingUpstairs &   0 & 437 &  12 &   2 &   0 &   0 \\ 
  WalkingDownstairs &   0 &   2 & 404 &   0 &   0 &   0 \\ 
  Sitting &   0 &   0 &   0 & 432 &  28 &   0 \\ 
  Standing &   0 &   0 &   0 &  57 & 504 &   0 \\ 
  Laying &   0 &   0 &   0 &   0 &   0 & 537 \\ 
   \hline
\end{tabular}
\end{table}

% latex table generated in R 3.2.2 by xtable 1.7-4 package
% Wed Nov  4 19:31:53 2015
\begin{table}[ht]
\centering
\caption{Model Statistics for Boosting Tree} 
\label{tab:conmat_stats_boost}
\begin{tabular}{rrrr}
  \hline
 & Sensitivity & Specificity & Balanced Accuracy \\ 
  \hline
Class: Walking & 0.98 & 0.99 & 0.99 \\ 
  Class: WalkingUpstairs & 0.96 & 0.99 & 0.97 \\ 
  Class: WalkingDownstairs & 0.92 & 1.00 & 0.96 \\ 
  Class: Sitting & 0.86 & 0.99 & 0.92 \\ 
  Class: Standing & 0.93 & 0.97 & 0.95 \\ 
  Class: Laying & 1.00 & 1.00 & 1.00 \\ 
   \hline
\end{tabular}
\end{table}


%\section{Code Listings}
%\lstinputlisting[label=lst:code, caption=Code Snippet, language=R]{../hw5.R}

\end{appendices}

\end{document}

% \input{.tex}

% \begin{figure}
%   \centering
%   \begin{subfigure}[b]{0.49\textwidth}
%     \caption{}
%     \includegraphics[width=\textwidth]{.pdf}
%     \label{fig:}
%   \end{subfigure}
%   \hfill
%   \begin{subfigure}[b]{0.49\textwidth}
%     \caption{}
%     \includegraphics[width=\textwidth]{.pdf}
%     \label{fig:}
%   \end{subfigure}
%   \caption{}
% \end{figure}

% \begin{figure}[!htb]
%   \centering
%   \caption{}
%   \includegraphics[scale=.5]{.pdf}
%   \label{fig:}
% \end{figure}

